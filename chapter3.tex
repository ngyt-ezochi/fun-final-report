\chapter{手法}	% TODO: 章題を記入.題は任意.
\thispagestyle{plain}   % chapterの直後に必ず指定

%TODO: 章の内容を記入.以下はサンプル.
この章では提案手法について述べる.
研究内容に応じ,提案する理論/仮説/モデル/アルゴリズム/システム/方法論/実装などについて説明する.
この部分が論文の主たる部分となる.章のタイトルはサンプルに縛られるものではなく,研究内容に応じて当然変わるものであるし,
章の数も,研究内容に応じて適切に設定すべきである.適切に担当教員からの指導を受けること.
以上を踏まえて,この章では,カレーライスの食べ方について,詳細に説明する.

\section{実験方法}
まず,スプーンを手に持つ.この際,落とさないようにしっかりと持つことが重要である.

\section{実験装置}
スプーンをカレー皿に挿入し,一口で食べられる適量をスプーンに載せる.

\section{分析方法}
スプーンをカレー皿から取り出し,口元まで運ぶ.掘削の際に過剰な量をスプーンに載せていると,この段階でスプーンからこぼれ落ちる可能性があるので注意が必要である.

