\chapter{関連研究}	% TODO: 章題を記入.題は任意.
\thispagestyle{plain}   % chapterの直後に必ず指定

%TODO: 章の内容を記入.以下はサンプル.
VRや触覚フィードバック,振動による触角フィードバックを用いたものとして, 多くの関連研究が存在する.
VRを活用した研究, 触覚フィードバックを活用して研究, その両方を組み合わせた研究について以下に述べる.

\section{仮想現実の活用}
VRを活用した関連研究としては, バイキングVR\cite{viking}がある.
バイキングVRとは9世紀のバイキングの野営地の風景や音を体験できるVRのことである.

この研究では, 文化遺産を対象として, 本物の情報を提供する魅力的なVR体験をデザインするためのアプローチについて説明している.
また, この研究では, VRの視覚的な情報に加えて, 再現した音の聴覚的な情報を利用している.
それによって, 当時の状況を再現しているのである.

\section{VRと触角フィードバック}
触覚フィードバックを活用した関連研究として, 2つの関連研究について述べる.

1つ目は, 弾性エネルギーによる触覚フィードバックと空中での打鍵およびスワイプ操作に対する有効性を検証した研究がある\cite{tactile}.
この論文では, エネルギーの充電段階で充電された弾性エネルギーを保存し, 刺激段階で指を打つ力を強化するためにそれらを放電するバネを備えたメカニズムによって
ショック刺激を強化する簡単で効果的な方法を提示している.
このメカニズムにより, 指先によりつけられた小型で軽量の刺激装置が開発され, 指が仮想オブジェクトに衝突したかのように感じるのに十分な強さの触覚フィードバックを生成することが実証されている.

2つ目は, 仮想オブジェクトの操作と探索のための指先の触覚デバイス\cite{figertip}についての研究がある.
VRでのオブジェクト操作中の没入感に対する主な障壁の1つは, 現実的な触覚フィードバックの欠如である.
この研究で開発したデバイスは, 3つの純粋な並進方向の自由度を持っている.
そのため, 重力, 摩擦, 剛性など, 物体を操作する際に多方向に作用する力を表現するのに適している.
この研究では, 被験者は仮想物体の重さの変化を知覚できるだけでなく, 操作する物体の質量に応じて握力を変化さすることができることがわかった.
さらに, 仮想物体の物理的特性を変化させることで, デバイス使用時のユーザーの知覚に全く影響を与えないことを示しました.

\section{VRと振動による触角フィードバック}
仮想現実と振動触覚フィードバックを用いた先行研究として, FingerVIPと呼ばれるモバイルバースの振動触覚フィードバックシステム\cite{vibration}を提案したものがある.
これでは, VRアプリケーションやゲームのデザイナーがターゲットの振動触覚フィードバックの適切な振動の構成を入力するための直感的で効率的な方法を提供している.

提案されたFingerVIPを利用してVRスポーツゲームで3種類の振動触覚フィードバックを設定し, FingerVIPがゲーム設計者の反復回数と振動の構成時間の削減に成功したことを検証している.
また, 本研究は3種類の振動触覚フィードバックとして, バスケットボールのキャッチ, ドリブル, シュートを行った.
受動的なキャッチに対しては, 大きな効果が見られた.しかし, 能動的なドリブル, シュートに関しては改善の余地が見られることがわかった.
\section{研究手法}
本研究のねらいは, 事前に収集(サンプリング)された触覚情報を、
振動でどこまで再現することができるのかを検証することである.
それを検証するための研究手法として, 以下の方法で研究を行う.


