\chapter{序論}
\thispagestyle{plain}

\section{背景}
% TODO: ここではVRがどのように使用されているのかについて触れたい
仮想現実(Virtual Reality:以下, VR)は, コンピュータによって作成された仮想的な空間を, あたかも現実であるかのように体験させる技術のことである.
この仮想空間を体験するためには, ヘッドマウントディスプレイ(Head Mounted Display:以下, HMD)と呼ばれるゴーグル型のデバイスを頭部に装着する必要がある場合が多い.
ゴーグル型の他にも, グラス型のデバイスを存在する.
近年では, Meta Platforms社傘下のフェイスブック・テクノロジーズが開発した「Oculus VR」を始め, HTC社の「VIVE PRO」, ソニー社の「Playstation VR」など数十種類の製品がある.
これに加え, スマートフォンをフロント部分に装着して使用すること場合もある.
VRは, 一般にはゲームなどの娯楽目的で使用されることが多い.
しかしながら, 教育, 医療, スポーツなどさまざまな場面で活用されることが増えている.

% TODO: 触覚フィードバックから振動を用いる研究のことについて言及する
VRデバイスでは, VRゴーグルだけではなく, コントローラーとを組み合わせて用いる場合が多い.
これらを使用することにより, 使用者は視覚的な情報だけではなく, 触覚的な情報からも仮想空間を現実であるかのように感じることができる.
触覚によるフィードバックを行う際に, 触覚として振動刺激を用いる場合が多い.

% TODO:ここではどんなことが先行研究では足りないのか. どんなことが問題であるのかについて述べる.
しかしながら, 

% TODO:ここではどんなことを明らかにすれば, 科学的に裏付けできるのかを考え, そこから目的を記述
\section{目的}
本研究では, 感圧センサを用いた仮想現実における触覚フィードバックについて研究する.
また, 使用者が感圧センサによって事前に収集(サンプリング)した現実の触覚を, 仮想オブジェクトの触覚で再現することを目的する.
% 本研究では, 「仮想現実における感圧センサを用いた触覚フィードバックデバイス」を開発し,
% 使用者が感圧センサによって事前に収集(サンプリング)した現実の触覚を, 仮想オブジェクトの触覚で再現することを目的とする.
% また, それに加え, 複数人に実験を行うことで個人によるデータの違いを検証する.
% このことから, 触覚フィードバックから得られる圧力再現の程度, 個人や物体による感覚の違いを検証することを研究目的とする.

% TODO:どういう研究をすることによって目的を成し遂げたいのかを記述
\section{検討方法}

% 何章で何をやるのか順番に列挙していく
\section{予告}
第2章では, 関連研究について解説する.
第3章では, 具体的研究手法について解説する.
第4章では, 研究によって得られた結果について解説する.
第5章では, 得られた結果をもとに〇〇について考察する.
第6章では, まとめとして本研究の総括を行う.

\section{本研究の位置付け}
VRにおける触覚フィードバックにサンプリングの考え方を導入することで,より簡便に視触覚VRシステムを構築することができる.
これは,安全で快適な情報社会の実現のために貢献する.