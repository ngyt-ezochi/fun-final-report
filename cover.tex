% TODO: 論文題目等の情報を以下に記入

%\newcommand{\jdoctitle}{修士論文}
%\newcommand{\edoctitle}{Master's Thesis}
\newcommand{\jtitle}{卒業論文日本語タイトル}  % 卒業論文の題名(日)
\newcommand{\etitle}{Title in English}   % 論文題目(英)
\newcommand{\jauthor}{野口 裕太}      % 著者名(日)
\newcommand{\eauthor}{Yuta Noguchi} % 著者名(英)
\newcommand{\jadvisor}{川嶋 稔夫}   % 指導教員名(日)
\newcommand{\eadvisor}{Toshio Kawashima}  % 指導教員名(英)
\newcommand{\jdate}{2022年1月25日}  % 論文提出日   (日)
\newcommand{\edate}{January 25th, 2022}  % 論文提出年月 (英)
\newcommand{\jkeywords}{仮想環境, 触覚フィードバック, キーワード3} % キーワード(日)
\newcommand{\ekeywords}{VR, Tactile feedback, Keyword3}   % キーワード(英)
\newcommand{\eshorttitle}{Your Short English Title Here}    % 短縮英題題名(おおよそ8 words以内)
\newcommand{\jdepartment}{情報アーキテクチャ学科}    % 学科名(日)
%\newcommand{\jdepartment}{複雑系知能学科}    % 学科名(日)
\newcommand{\jcourse}{情報システムコース}    % コース名(日)
%\newcommand{\jcourse}{高度ICTコース}    % コース名(日)
%\newcommand{\jcourse}{情報デザインコース}    % コース名(日)
%\newcommand{\jcourse}{複雑系コース}    % コース名(日)
%\newcommand{\jcourse}{知能システムコース}    % コース名(日)
\newcommand{\studentID}{1018100}    % 学籍番号
\newcommand{\edepartment}{Department of Media Architecture}    % 学科名(英)
%\newcommand{\edepartment}{Department of Complex and Intelligent Systems}    % 学科名(英)
\newcommand{\ecourse}{Information Systems Course}    % コース名(英)
%\newcommand{\ecourse}{Advanced ICT Course}    % コース名(英)
%\newcommand{\ecourse}{Information Design Course}    % コース名(英)
%\newcommand{\ecourse}{Complex Systems Course}    % コース名(英)
%\newcommand{\ecourse}{Intelligent Systems Course}    % コース名(英)
