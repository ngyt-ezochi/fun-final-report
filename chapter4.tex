\chapter{実験と評価}	% TODO: 章題を記入.題は任意.
\thispagestyle{plain}   % chapterの直後に必ず指定

%TODO: 章の内容を記入.以下はサンプル.
%TODO: 実験の結果は主観を交えず、得られた結果を整理して客観的に示す
この章では本研究で行った実験と評価および考察について述べる.
研究内容によっては,考察は独立の章に分けたほうが適切なことも多い.
また,実験と評価と考察で節を分けなければならないというものでもない.
自らの研究内容を論文にまとめるにあたって,最も適切な方法を選択することが重要である.
それはそれとして,この章では,数式の書き方と,参考文献のリスト法
について記述する.
研究分野によっては慣習が異なることがあるので,適切に担当教員からの指導を受けること.
%出典は昨年度までのWordテンプレートである.

\section{触角デバイスの試作}
\subsection{触角センサ}
\subsection{デバイスの特性}

\section{触覚データのサンプリング}
\subsection{振動による触角の再現方法}
\subsection{VRと振動に触角の再現方法}
\subsection{再現方法}
\subsection{タイミングを変えた再現方法}