% # 公立はこだて未来大学・卒業論文テンプレートファイル(unicode)
%
% ## 改訂履歴:
% - 2019/11/18 修士論文テンプレート初版 作成者:三上貞芳
% - 2019/12/05 卒業論文用に改変:寺沢憲吾
% - 2021/11/29 詳細調整:中小路久美代
%
% ## 論文作成の手順
%
% 1. 以下のtexファイルを作成してください
% - cover.tex           氏名・タイトル等の表紙情報
% - eabstract.tex       英語アブストラクト
% - jabstract.tex       日本語アブストラクト
% - chapterX.tex        本文第X章
% - publications.tex    発表・採録等の実績(確定分も含む)←※卒論では必須としない
% - acknowledgment.tex  謝辞
% - bibliography.tex    参考文献
%
% 2. このテンプレートの「TODO: 本文」以下に,作成した章に対応する\input{chapterX.tex}文を追記してください(Xは章番号).付録の場合は「TODO: 付録」以下に追記してください.
% 2-1. 「TODO: 付録」の部分は,必要がなければ削除してください(大半の学生は必要がないと思います)
% 2-2. 「図表一覧等自動生成」は,初期設定ではオフにしてあります.必要があればオンにしてください(コメント化を解除してください).
%
% 3. このテンプレートをuplatex環境でコンパイルし,PDFを作成します.Overleaf環境においては、コンパイラを「LaTeX」としてください.
%

\documentclass[uplatex, a4paper, report, 11pt, oneside]{jsbook}

% packages
\usepackage[utf8]{inputenc}
\usepackage[dvipdfmx]{graphicx}
\usepackage{lmodern}             % use latin modern font
\usepackage{amsmath,amssymb,amsthm}

\usepackage{layout}

% 未来大書式設定
% % # 公立はこだて未来大学・修士論文書式定義ファイル
%
% (https://github.com/kmiya/naist-thesis-tmpl を一部参照)
% 
% ## 改訂履歴:
% - 2019/11/18 初版 作成者:三上貞芳

% ## 使用法;
% - main.texを参照してください.
% - **このファイルを変更する必要はありません**

\usepackage[dvipdfmx]{graphicx}
\usepackage[utf8]{inputenc}
\usepackage[T1]{fontenc}
\usepackage{lmodern}
\usepackage{amsmath,amssymb,amsthm}
\let\equation\gather
\let\endequation\endgather
\usepackage{fancybox}
\usepackage[flushmargin,symbol]{footmisc}
\usepackage[nottoc]{tocbibind}
\usepackage[dvipdfmx,%
 bookmarks=true,%
 bookmarksnumbered=true,%
 setpagesize=false,%
 colorlinks=false,%
 linkbordercolor={0.8 0.8 0.8},%
 citebordercolor={0.8 0.8 0.8},%
 pdfborder={0 0 0.6},%
% urlcolor=black,linkcolor=black,citecolor=black,%
 pdftitle={},% 修論のタイトルを入れる
 pdfauthor={},% 名前を入れる
 pdfsubject={Master's thesis},%
 pdfkeywords={\ekeywords}]{hyperref}
\usepackage{pxjahyper}

% ページレイアウト
\textheight=20.6truecm          % 縦
\textwidth=14.5truecm           % 横
\oddsidemargin=0.6truecm        % 左マージン(1inオフセット後)
\evensidemargin=-3.8truecm      % 右マージン(1inオフセット後)

% フォント等調整
% 参考文献
\def\thebibliography#1{\chapter*{参考文献\markboth
 {参 考 文 献}{参 考 文 献}\addcontentsline{toc}{chapter}{参考文献}}\list
 {[\arabic{enumi}]}{\settowidth\labelwidth{[#1]}\leftmargin\labelwidth
 \advance\leftmargin\labelsep
 \usecounter{enumi}}
 \def\newblock{\hskip .11trueem plus .33trueem minus -.07trueem}
 \sloppy
 \sfcode`\.=1000\relax}
\let\endthebibliography=\endlist

% 章
\makeatletter%%
\def\@makechapterhead#1{\hbox{}%
  \vskip-1\Cvs
  {\parindent\z@
%  \reset@font\LARGE\bfseries
   \raggedright\reset@font\Large\bfseries% 左揃え
   \ifnum \c@secnumdepth >\m@ne
     \setlength\@tempdima{\linewidth}%
     \vtop{\hsize\@tempdima%
         \@chapapp\thechapter\@chappos\mbox{\ \ }%
     #1}%
   \else
     #1\relax
   \fi}\nobreak\vskip1\Cvs}
\makeatother%%

\makeatletter%%
\def\@makeschapterhead#1{\hbox{}%
  \vskip-1\Cvs
  {\parindent \z@ \raggedright
    \normalfont
    \interlinepenalty\@M
    \Large\headfont #1\par\nobreak
    \vskip1\Cvs}}
\makeatother%%

% 節
\makeatletter%%
\renewcommand{\section}{%
  \@startsection{section}% #1 見出し
   {1}% #2 見出しのレベル
   {\z@}% #3 横組みの場合,見出し左の空き(インデント量)
   {1.5\Cvs \@plus.5\Cdp \@minus.2\Cdp}% #4 見出し上の空き
   {.5\Cvs \@plus.3\Cdp}% #5 見出し下の空き (負の値なら見出し後の空き)
  {\raggedright\reset@font\large\bfseries}% 左揃え
}%
\makeatother%%

% 小節
\makeatletter%%
\renewcommand{\subsection}{%
  \@startsection{subsection}% #1 見出し
   {1}% #2 見出しのレベル
   {\z@}% #3 横組みの場合,見出し左の空き(インデント量)
   {1.5\Cvs \@plus.5\Cdp \@minus.2\Cdp}% #4 見出し上の空き
   {.5\Cvs \@plus.3\Cdp}% #5 見出し下の空き (負の値なら見出し後の空き)
  {\raggedright\reset@font\normalsize\bfseries}% 左揃え
}%
\makeatother%%

% 表題
\makeatletter
\def\@startsection#1#2#3#4#5#6{%
  \if@noskipsec \leavevmode \fi
  \par
  \@tempskipa #4\relax
  \if@english \@afterindentfalse \else \@afterindenttrue \fi
  \ifdim \@tempskipa <\z@
    \@tempskipa -\@tempskipa \@afterindentfalse
  \fi
  \if@nobreak
    \everypar{}%
  \else
    \addpenalty\@secpenalty
    \ifdim \@tempskipa >\z@
      \vskip\@tempskipa
      \if@slide\else
        \null
        \vspace{-\baselineskip}%
      \fi
    \fi
  \fi
  \noindent
  \@ifstar
    {\@ssect{#3}{#4}{#5}{#6}}%
    {\@dblarg{\@sect{#1}{#2}{#3}{#4}{#5}{#6}}}}
\makeatother

% 式番号
\makeatletter
  \renewcommand{\theequation}{%
  \thesection.\arabic{equation}}
    \@addtoreset{equation}{section}
\makeatother


% 図番号
\makeatletter
 \renewcommand{\thefigure}{%
  \thechapter.\arabic{figure}}
   \@addtoreset{figure}{chapter}
 \makeatother
\makeatletter

% 目次に小節を表示
\setcounter{tocdepth}{4}


% TODO: タイトル・著者等の情報
% TODO: 論文題目等の情報を以下に記入

%\newcommand{\jdoctitle}{修士論文}
%\newcommand{\edoctitle}{Master's Thesis}
\newcommand{\jtitle}{卒業論文日本語タイトル}  % 卒業論文の題名(日)
\newcommand{\etitle}{Title in English}   % 論文題目(英)
\newcommand{\jauthor}{野口 裕太}      % 著者名(日)
\newcommand{\eauthor}{Yuta Noguchi} % 著者名(英)
\newcommand{\jadvisor}{川嶋 稔夫}   % 指導教員名(日)
\newcommand{\eadvisor}{Toshio Kawashima}  % 指導教員名(英)
\newcommand{\jdate}{2022年1月25日}  % 論文提出日   (日)
\newcommand{\edate}{January 25th, 2022}  % 論文提出年月 (英)
\newcommand{\jkeywords}{キーワード1, キーワード2, キーワード3} % キーワード(日)
\newcommand{\ekeywords}{Keyword1, Keyword2, Keyword3}   % キーワード(英)
\newcommand{\eshorttitle}{Your Short English Title Here}    % 短縮英題題名(おおよそ8 words以内)
\newcommand{\jdepartment}{情報アーキテクチャ学科}    % 学科名(日)
%\newcommand{\jdepartment}{複雑系知能学科}    % 学科名(日)
\newcommand{\jcourse}{情報システムコース}    % コース名(日)
%\newcommand{\jcourse}{高度ICTコース}    % コース名(日)
%\newcommand{\jcourse}{情報デザインコース}    % コース名(日)
%\newcommand{\jcourse}{複雑系コース}    % コース名(日)
%\newcommand{\jcourse}{知能システムコース}    % コース名(日)
\newcommand{\studentID}{1018100}    % 学籍番号
\newcommand{\edepartment}{Department of Media Architecture}    % 学科名(英)
%\newcommand{\edepartment}{Department of Complex and Intelligent Systems}    % 学科名(英)
\newcommand{\ecourse}{Information Systems Course}    % コース名(英)
%\newcommand{\ecourse}{Advanced ICT Course}    % コース名(英)
%\newcommand{\ecourse}{Information Design Course}    % コース名(英)
%\newcommand{\ecourse}{Complex Systems Course}    % コース名(英)
%\newcommand{\ecourse}{Intelligent Systems Course}    % コース名(英)


% TODO: 英語アブストラクト
% TODO: 英文アブストラクトを以下の{}内に記述(以下はダミーテキスト)
\newcommand{\eabstract}{

(Abstract should be about 150--200 words. Following is a sample text.)
Lorem ipsum dolor sit amet, consectetuer adipiscing elit. Maecenas porttitor congue massa. Fusce posuere, magna sed pulvinar ultricies, purus lectus malesuada libero, sit amet commodo magna eros quis urna.
Nunc viverra imperdiet enim. Fusce est. Vivamus a tellus.
Pellentesque habitant morbi tristique senectus et netus et malesuada fames ac turpis egestas. Proin pharetra nonummy pede. Mauris et orci.
Aenean nec lorem. In porttitor. Donec laoreet nonummy augue.
Suspendisse dui purus, scelerisque at, vulputate vitae, pretium mattis, nunc. Mauris eget neque at sem venenatis eleifend. Ut nonummy.
Lorem ipsum dolor sit amet, consectetuer adipiscing elit. Maecenas porttitor congue massa. Fusce posuere, magna sed pulvinar ultricies, purus lectus malesuada libero, sit amet commodo magna eros quis urna.
Nunc viverra imperdiet enim. Fusce est. Vivamus a tellus.
Pellentesque habitant morbi tristique senectus et netus et malesuada fames ac turpis egestas. Proin pharetra nonummy pede. Mauris et orci.
Aenean nec lorem. In porttitor. Donec laoreet nonummy augue.
Suspendisse dui purus, scelerisque at, vulputate vitae, pretium mattis, nunc. Mauris eget neque at sem venenatis eleifend. Ut nonummy.
}

% TODO: 日本語アブストラクト
% TODO: 日本語アブストラクトを以下の{}内に記述(以下はダミーテキスト)
\newcommand{\jabstract}{

(概要は約400字とすること.以下はダミーテキスト)
いろはにほへとちりぬるをわかよたれそつねならむういのおくやまけふこえてあさきゆめみしえひもせす.
いろはにほへとちりぬるをわかよたれそつねならむういのおくやまけふこえてあさきゆめみしえひもせす.
いろはにほへとちりぬるをわかよたれそつねならむういのおくやまけふこえてあさきゆめみしえひもせす.
いろはにほへとちりぬるをわかよたれそつねならむういのおくやまけふこえてあさきゆめみしえひもせす.
いろはにほへとちりぬるをわかよたれそつねならむういのおくやまけふこえてあさきゆめみしえひもせす.
いろはにほへとちりぬるをわかよたれそつねならむういのおくやまけふこえてあさきゆめみしえひもせす.
いろはにほへとちりぬるをわかよたれそつねならむういのおくやまけふこえてあさきゆめみしえひもせす.

}

% page size
\textheight     = 22.6truecm
\textwidth      = 14.7truecm
\oddsidemargin  = 0.6truecm

% header and footer
\usepackage{fancyhdr}
\pagestyle{fancy}
\setlength{\footskip}{16pt}
\fancyhf{}
\renewcommand{\chaptermark}[1]{\markboth{\thechapter.\ #1}{}}
\rhead{\leftmark}
\renewcommand{\headrulewidth}{0pt}
\cfoot{\thepage}
\lfoot{~~ \\BA thesis, Future University Hakodate}
\lhead{\eshorttitle}

%-------------------------------------
\begin{document}

\thispagestyle{empty}
\vspace*{4truemm}
\begin{center}
    \LARGE\bfseries
    卒業論文
\end{center}
\vspace*{2truemm}
\begin{center}
    \LARGE\bfseries\jtitle
\end{center}
\vspace*{1em}
\begin{center}
    \large\bfseries 公立はこだて未来大学\par%
    システム情報科学部~~\jdepartment\par%
    \jcourse~~\studentID
\end{center}
\vspace*{1em}
\begin{center}
    \Large\bfseries\jauthor
\end{center}
\vspace*{1em}
\begin{center}
    \large 指導教員~~~~\jadvisor\par
    \vspace{0.5em}
    \large 提出日~~~~\jdate
\end{center}
\vspace*{3em}
\begin{center}
\textbf{\Large BA Thesis}\par
\vspace*{2em}
\textbf{\Large \etitle}\par
\vspace*{1em}
{\normalsize by}\par
\vspace*{1em}
{\large \eauthor}\par
\vspace*{1.5em}
\ecourse, \edepartment \par
School of Systems Information Science, Future University Hakodate

% \vspace*{0.5em}
\normalsize Supervisor: \quad \eadvisor \par
\vspace*{2em}
Submitted on \edate
\end{center}
%\vspace*{\fill}

% 英語アブストラクト作成
\clearpage
\thispagestyle{empty}
%\vspace*{30truemm}
\noindent
\textbf{Abstract--}~
\eabstract

\vspace*{1em}
\noindent
\textbf{Keywords:}~
\ekeywords

% 日本語アブストラクト作成
%\newpage
%\thispagestyle{empty}
\vspace*{20truemm}
\noindent
\textgt{概~要:}~
\jabstract

\vspace*{1em}
\noindent
\textgt{キーワード:}~
\jkeywords

% 目次
\thispagestyle{empty}
\pagestyle{empty}
\tableofcontents

% ページ番号初期化
\setcounter{page}{0}

% TODO: 本文
\chapter{序論}
\thispagestyle{plain}

\section{背景}
% TODO: ここではVRがどのように使用されているのかについて触れたい
仮想現実(Virtual Reality:以下, VR)は, コンピュータによって作成された仮想的な空間を, あたかも現実であるかのように体験させる技術のことである.
この仮想空間を体験するためには, ヘッドマウントディスプレイ(Head Mounted Display:以下, HMD)と呼ばれるゴーグル型のデバイスを頭部に装着する必要がある場合が多い.
ゴーグル型の他にも, グラス型のデバイスを存在する.
近年では, Meta Platforms社傘下のフェイスブック・テクノロジーズが開発した「Oculus VR」を始め, HTC社の「VIVE PRO」, ソニー社の「Playstation VR」など数十種類の製品がある.
これに加え, スマートフォンをフロント部分に装着して使用すること場合もある.
VRは, 一般にはゲームなどの娯楽目的で使用されることが多い.
しかしながら, 教育, 医療, スポーツなどさまざまな場面で活用されることが増えている.

% TODO: 触覚フィードバックから振動を用いる研究のことについて言及する
VRデバイスでは, VRゴーグルだけではなく, コントローラーとを組み合わせて用いる場合が多い.
これらを使用することにより, 使用者は視覚的な情報だけではなく, 触覚的な情報からも仮想空間を現実であるかのように感じることができる.
触覚によるフィードバックを行う際に, 触覚として振動刺激を用いる場合が多い.

% TODO:ここではどんなことが先行研究では足りないのか. どんなことが問題であるのかについて述べる.
しかしながら, 

% TODO:ここではどんなことを明らかにすれば, 科学的に裏付けできるのかを考え, そこから目的を記述
\section{目的}
本研究では, 感圧センサを用いた仮想現実における触覚フィードバックについて研究する.
また, 使用者が感圧センサによって事前に収集(サンプリング)した現実の触覚を, 仮想オブジェクトの触覚で再現することを目的する.
% 本研究では, 「仮想現実における感圧センサを用いた触覚フィードバックデバイス」を開発し,
% 使用者が感圧センサによって事前に収集(サンプリング)した現実の触覚を, 仮想オブジェクトの触覚で再現することを目的とする.
% また, それに加え, 複数人に実験を行うことで個人によるデータの違いを検証する.
% このことから, 触覚フィードバックから得られる圧力再現の程度, 個人や物体による感覚の違いを検証することを研究目的とする.

% TODO:どういう研究をすることによって目的を成し遂げたいのかを記述
\section{検討方法}

% 何章で何をやるのか順番に列挙していく
\section{予告}
第2章では, 関連研究について解説する.
第3章では, 具体的研究手法について解説する.
第4章では, 研究によって得られた結果について解説する.
第5章では, 得られた結果をもとに〇〇について考察する.
第6章では, まとめとして本研究の総括を行う.

\section{本研究の位置付け}
VRにおける触覚フィードバックにサンプリングの考え方を導入することで,より簡便に視触覚VRシステムを構築することができる.
これは,安全で快適な情報社会の実現のために貢献する.
\chapter{関連研究}	% TODO: 章題を記入.題は任意.
\thispagestyle{plain}   % chapterの直後に必ず指定

%TODO: 章の内容を記入.以下はサンプル.
VRや触覚フィードバック,振動による触角フィードバックを用いたものとして, 多くの関連研究が存在する.
VRを活用した研究, 触覚フィードバックを活用して研究, その両方を組み合わせた研究について以下に述べる.

\section{仮想現実の活用}
VRを活用した関連研究としては, バイキングVR\cite{viking}がある.
バイキングVRとは9世紀のバイキングの野営地の風景や音を体験できるVRのことである.

この研究では, 文化遺産を対象として, 本物の情報を提供する魅力的なVR体験をデザインするためのアプローチについて説明している.
また, この研究では, VRの視覚的な情報に加えて, 再現した音の聴覚的な情報を利用している.
それによって, 当時の状況を再現しているのである.

\section{VRと触角フィードバック}
触覚フィードバックを活用した関連研究として, 2つの関連研究について述べる.

1つ目は, 弾性エネルギーによる触覚フィードバックと空中での打鍵およびスワイプ操作に対する有効性を検証した研究がある\cite{tactile}.
この論文では, エネルギーの充電段階で充電された弾性エネルギーを保存し, 刺激段階で指を打つ力を強化するためにそれらを放電するバネを備えたメカニズムによって
ショック刺激を強化する簡単で効果的な方法を提示している.
このメカニズムにより, 指先によりつけられた小型で軽量の刺激装置が開発され, 指が仮想オブジェクトに衝突したかのように感じるのに十分な強さの触覚フィードバックを生成することが実証されている.

2つ目は, 仮想オブジェクトの操作と探索のための指先の触覚デバイス\cite{figertip}についての研究がある.
VRでのオブジェクト操作中の没入感に対する主な障壁の1つは, 現実的な触覚フィードバックの欠如である.
この研究で開発したデバイスは, 3つの純粋な並進方向の自由度を持っている.
そのため, 重力, 摩擦, 剛性など, 物体を操作する際に多方向に作用する力を表現するのに適している.
この研究では, 被験者は仮想物体の重さの変化を知覚できるだけでなく, 操作する物体の質量に応じて握力を変化さすることができることがわかった.
さらに, 仮想物体の物理的特性を変化させることで, デバイス使用時のユーザーの知覚に全く影響を与えないことを示しました.

\section{VRと振動による触角フィードバック}
仮想現実と振動触覚フィードバックを用いた先行研究として, FingerVIPと呼ばれるモバイルバースの振動触覚フィードバックシステム\cite{vibration}を提案したものがある.
これでは, VRアプリケーションやゲームのデザイナーがターゲットの振動触覚フィードバックの適切な振動の構成を入力するための直感的で効率的な方法を提供している.

提案されたFingerVIPを利用してVRスポーツゲームで3種類の振動触覚フィードバックを設定し, FingerVIPがゲーム設計者の反復回数と振動の構成時間の削減に成功したことを検証している.
また, 本研究は3種類の振動触覚フィードバックとして, バスケットボールのキャッチ, ドリブル, シュートを行った.
受動的なキャッチに対しては, 大きな効果が見られた.しかし, 能動的なドリブル, シュートに関しては改善の余地が見られることがわかった.
\section{研究手法}
本研究のねらいは, 事前に収集(サンプリング)された触覚情報を、
振動でどこまで再現することができるのかを検証することである.
それを検証するための研究手法として, 以下の方法で研究を行う.



\chapter{手法}	% TODO: 章題を記入.題は任意.
\thispagestyle{plain}   % chapterの直後に必ず指定

%TODO: 章の内容を記入.以下はサンプル.
この章では提案手法について述べる.
研究内容に応じ,提案する理論/仮説/モデル/アルゴリズム/システム/方法論/実装などについて説明する.
この部分が論文の主たる部分となる.章のタイトルはサンプルに縛られるものではなく,研究内容に応じて当然変わるものであるし,
章の数も,研究内容に応じて適切に設定すべきである.適切に担当教員からの指導を受けること.
以上を踏まえて,この章では,カレーライスの食べ方について,詳細に説明する.

\section{実験方法}
まず,スプーンを手に持つ.この際,落とさないようにしっかりと持つことが重要である.

\section{実験装置}
スプーンをカレー皿に挿入し,一口で食べられる適量をスプーンに載せる.

\section{分析方法}
スプーンをカレー皿から取り出し,口元まで運ぶ.掘削の際に過剰な量をスプーンに載せていると,この段階でスプーンからこぼれ落ちる可能性があるので注意が必要である.


\chapter{実験と評価および考察}	% TODO: 章題を記入.題は任意.
\thispagestyle{plain}   % chapterの直後に必ず指定

%TODO: 章の内容を記入.以下はサンプル.
この章では本研究で行った実験と評価および考察について述べる.
研究内容によっては,考察は独立の章に分けたほうが適切なことも多い.
また,実験と評価と考察で節を分けなければならないというものでもない.
自らの研究内容を論文にまとめるにあたって,最も適切な方法を選択することが重要である.
それはそれとして,この章では,数式の書き方と,参考文献のリスト法
について記述する.
研究分野によっては慣習が異なることがあるので,適切に担当教員からの指導を受けること.
%出典は昨年度までのWordテンプレートである.

\section{触角デバイスの試作}
\subsection{触角センサ}
\subsection{デバイスの特性}

\section{触覚データのサンプリング}
\subsection{振動による触角の再現方法}
\subsection{VRと振動に触角の再現方法}
\subsection{再現方法}
\subsection{タイミングを変えた再現方法}
\chapter{考察}	% TODO: 章題を記入.題は任意.
\thispagestyle{plain}   % chapterの直後に必ず指定

%TODO: この研究で最も言いたかった主張を記述
この章では研究で得られた血子をもとに、考察について述べる.
研究内容によっては,考察は独立の章に分けたほうが適切なことも多い.
また,実験と評価と考察で節を分けなければならないというものでもない.
自らの研究内容を論文にまとめるにあたって,最も適切な方法を選択することが重要である.
それはそれとして,この章では,数式の書き方と,参考文献のリスト法
について記述する.
研究分野によっては慣習が異なることがあるので,適切に担当教員からの指導を受けること.
%出典は昨年度までのWordテンプレートである.

\section{触角デバイスの試作}
\subsection{触角センサ}
\subsection{デバイスの特性}
\chapter{結論}
\thispagestyle{plain}

この章は最終章である.
第1章と最終章は対比がとれていることが望ましい.
具体的には,「序論」ではじめたのなら「結論」で終わり,
「はじめに」ではじめたのなら「おわりに」で終わる.
「緒言」ではじめたのなら「結言」で終わる.

\section{句読点}
日本語の文書で一般に用いられる読点には「、」「,」の2種類があり,
句点には「。」「.」の2種類がある.
情報系では「,.」を用いることが多いが,
どちらを用いるべきかは分野の慣習により異なることがあるので,指導教員の指示に従うこと.
いずれにしても,両者が無秩序に混在しているのは悪い文書である.

\section{まとめ}
論文の執筆法は,研究分野によりさまざまなルールや慣習がある.
また,研究内容に応じ,最適な章立てや叙述の順序なども異なってくる.
このスタイルファイルに書かれている内容はあくまで例にすぎない.
実際に論文を執筆し,提出する際は,担当教員の指導に従うこと.
また,論文の書き方や研究の進め方を指南する書籍やウェブサイトは多数存在するので,
適宜参照すると良い.
この場合も,分野によって論文の書き方や研究の進め方が異なることはあるので,
担当教員の指導を受けることが望ましい.

% 以下必要に応じてchapterX.texを作成してinput文を記入

% TODO: 謝辞
\pagestyle{plain}
\chapter*{謝辞}
% TODO: 謝辞を以下に記入

謝辞を記入する.


% TODO: 発表等実績
% \chapter*{発表・採録実績}

% TODO: 発表・採録実績(確定分も含む)を以下の例のように記入

\subsection*{発表等}
\begin{enumerate}
\renewcommand{\labelenumi}{[\arabic{enumi}]}
    \item 発表その1
    \item 発表その2
    \item 発表予定(発表予定年月)
\end{enumerate}

\subsection*{学術論文,国際会議等(査読付き)}
\begin{enumerate}
\renewcommand{\labelenumi}{[\arabic{enumi}]}
    \item 論文その1
    \item 国際会議その1
    \item 採録決定論文(採録予定年月)
\end{enumerate}


% TODO: 参考文献
% TODO: 参考文献を以下のように記入.

\begin{thebibliography}{99}
 \bibitem{item1}
Reference1
 \bibitem{item2}
Reference2
\end{thebibliography}


% TODO: 付録.必要がなければ削除すること
\appendix
\chapter*{付録}	% TODO: 章題を記入.題は任意.
\thispagestyle{plain}   % chapterの直後に必ず指定

%TODO: 章の内容を記入.以下はサンプル.
プログラムのソースリスト,その他関連資料などを,【必要があれば】載せる.
必要ない場合は,このページごと削除すること.
\TeX の場合は main.tex 内の \yen appendix 以下の2行を削除(またはコメント化)すればよい.
Wordの場合は前のページの「改ページ」以降を削除すればよい.


% 図表一覧等自動生成
%\listoffigures
%\thispagestyle{plain}
%\listoftables
%\thispagestyle{plain}


\end{document}
